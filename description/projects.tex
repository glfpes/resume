\resheading{项目经历}
  \begin{itemize}[leftmargin=*]
    \item
      \ressubsingleline{混合云虚拟机例调度}{Python/openstack开发}{2015.09 -- 至今}
      {\small
      \begin{itemize}
        \item 开发环境: openstack, linux, pycharm,django
        \item 项目简介: 实现一个混合云调度系统,根据异构云的当前状态及用户给出的需求,选用最佳的调度策略,将用户申请的实例分配到最合适的云内。主要模块包括后台调度器与前端web界面及针对各个异构云的adapter驱动。
        项目隶属于课题: 基于中国云产品的混合云关键技术与系统$\,$(国家863计划)。
        \item 职责任务: 
	        \subitem 负责搭建并维护openstack公有云。
	        \subitem 学习openstack的nova组件的调度器设计模式,给出我们的调度器设计策略。
	        \subitem 利用Python网络编程,实现满足需求的调度器,并实现调度器与adapter驱动及web界面的通信。
      \end{itemize}
      }
  \item
  \ressubsingleline{基于Spark的运维数据自动化分析系统}{Spark/Python}{2016.05 -- 2016.09}
  {\small
     	\begin{itemize}
     		\item 开发环境: spark, scala, python, linux, javascript
     		\item 项目简介: 
     		1,根据需求给定的指标开发基于python的虚拟机健康信息数据搜集脚本,定期将搜集的数据发送到指定的Rest接口。
     		2,在公司的hadoop平台上开发spark分析程序,基于需求的阈值等指标分析各个虚拟机的健康情况。
     		3,在虚机管理网站上编写javascript脚本以及相应的javabean,来图形化显示虚拟机的健康情况。
     		% 项目使用 Java 语言开发,并采用了 SpringMVC 和 JPA 开源技术框架。
     		% 项目隶属于课题: 支撑区域和地方支柱产业的制造业信息化综合应用示范$\,$(国家支撑计划)。
     		\item 职责任务: 参与团队开发,按照feature来迭代任务。
     	\end{itemize}
  }
    \item
      \ressubsingleline{基于Rolling Shutter效应的室内定位}{C++/系统全栈开发}{2015.01 -- 2016.05}
      {\small
      \begin{itemize}
        \item 开发环境: opencv,arduino,python,android, php 
        \item 项目简介: 提出一种基于可见光的室内定位方法,结合手机的内置传感器,实现可靠的室内定位。实现方面,需要arduino编程控制光源编码,C++结合opencv编程实现解码,python脚本进行定位,android编程利用手机传感器来跟踪用户。
        % 项目使用 Java 语言开发,并采用了 SpringMVC 和 JPA 开源技术框架。
        % 项目隶属于课题: 支撑区域和地方支柱产业的制造业信息化综合应用示范$\,$(国家支撑计划)。
        \item 职责任务: 独自负责整个项目的软硬件各个组件。
      \end{itemize}
      }
    \item
    \ressubsingleline{网络性能测量}{C++/网络开发}{2014.10 -- 2015.06}
    {\small
    \begin{itemize}
      \item 开发环境: linux, icmp
      \item 项目简介: 实现一种网络性能测量方式,对已知的多个节点间的网络性能参数进行主动测量和被动测量。
      % 该项目使用 Java 语言,采用 Struts2、Spring、Hibernate 框架开发。
      \item 职责任务: 使用icmp方式完成各个节点间两两的网络可用性,带宽,延迟,抖动,丢包率,拥塞情况的测量。
    \end{itemize}
    }
    \item
    \ressubsingleline{基于kerberos的物联网授权认证协议}{Python/Restful}{2011.04 -- 2011.09}
    {\small
    \begin{itemize}
      \item 开发环境: linux,eclipse,django
      \item 项目简介: 基于kerberos协议的方法,实现一种适合于物联网环境下的对终端的身份认证与授权管理协议,并完成一个协议的demo。
      \item 职责任务: 结合google的openid授权方式,对kerberos各个认证服务器的职责进行重新规划,并重新设计的一套可靠且能切实有效的降低终端的计算压力与通信压力的加密解密流程,使得改良的协议适合于物联网环境下的终端。 同时使用python与restful通信方式,实现一个demo,完成整套认证授权协议的demo。
    \end{itemize}
    }
  \end{itemize}